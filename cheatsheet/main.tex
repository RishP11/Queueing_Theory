\documentclass{article}
\usepackage[landscape]{geometry}
\usepackage{url}
\usepackage{multicol}
\usepackage{amsmath}
\usepackage{amsfonts}
\usepackage{tikz}
\usetikzlibrary{decorations.pathmorphing,shadings} % Include the shadings library

\usepackage{colortbl}
\usepackage{xcolor}
\usepackage{mathtools}
\usepackage{amsmath,amssymb}
\usepackage{enumitem}

\title{Basics of Queueing Theory}
\usepackage[utf8]{inputenc}

\advance\topmargin-.8in
\advance\textheight3in
\advance\textwidth3in
\advance\oddsidemargin-1.5in
\advance\evensidemargin-1.5in
\parindent0pt
\parskip2pt
\newcommand{\hr}{\centerline{\rule{3.5in}{1pt}}}

% Define colors sampled from the magma colormap
\definecolor{magma1}{rgb}{0.001462, 0.000466, 0.013866}
\definecolor{magma2}{rgb}{0.316654, 0.07169, 0.48538}
\definecolor{magma3}{rgb}{0.716387, 0.214982, 0.47529}
\definecolor{magma4}{rgb}{0.9867, 0.535582, 0.38221}
\definecolor{magma5}{rgb}{0.987053, 0.991438, 0.749504}

\begin{document}

\begin{center}{\huge{\textbf{Basics of Queueing Theory}}}\\
{\large By Rishabh Pomaje}
\end{center}
\begin{multicols*}{2}

% Define the custom magma gradient shading
\tikzstyle{mybox} = [draw=black, fill=white, very thick,
    rectangle, rounded corners, inner sep=10pt, inner ysep=10pt]
\tikzstyle{fancytitle} = [
    shade,
    left color=magma2,
    right color=magma4,
    text=white,
    font=\bfseries
]

%------------ CONTEÚDO CAIXA RANDOM ---------------
\begin{tikzpicture}
\node [mybox] (box){%
    \begin{minipage}{0.3\textwidth}
        1. Assumptions:
        \begin{enumerate}
            \item The arrivals are Poisson random process with rate $\lambda$.
            \item The service times are Exponentially distributed with rate $\mu\ (> \lambda)$.
            \item There is a single server with no limit on the queue size.
        \end{enumerate}
        2. The distribution of the state at time $t$ is given by,
    \end{minipage}
};
%------------ CAIXA RANDOM ---------------------
\node[fancytitle, right=10pt] at (box.north west) {$M/M/1$ Queue};
\end{tikzpicture}

%------------ CONTEÚDO CAIXA MatPlotLib ---------------
\begin{tikzpicture}
\node [mybox] (box){%
    \begin{minipage}{0.3\textwidth}
        Para usar a biblioteca MatPlotLib, comece importando estes módulos Python: \\
        \\
        \textit{import numpy as np} \\
        \textit{import pandas as pd} \\
        \textit{from pandas import DataFrame, Series} \\
        \textit{import matplotlib.pyplot as plt} \\
        \textit{import matplotlib} \\
        \\
        {\bf Pyplot} é uma coleção de funções no estilo de comandos que fazem a biblioteca matplotlib funcionar como o MatLab. Cada função pyplot faz alguma alteração na plotagem do gráfico.
    \end{minipage}
};
%------------ CAIXA PRELIMINARES ---------------------
\node[fancytitle, right=10pt] at (box.north west) {$M/M/1/N$ Queue};
\end{tikzpicture}

% Add other TikZ boxes as needed...
\begin{tikzpicture}
    \node [mybox] (box){%
        \begin{minipage}{0.3\textwidth}
            Para usar a biblioteca MatPlotLib, comece importando estes módulos Python: \\
            \\
            \textit{import numpy as np} \\
            \textit{import pandas as pd} \\
            \textit{from pandas import DataFrame, Series} \\
            \textit{import matplotlib.pyplot as plt} \\
            \textit{import matplotlib} \\
            \\
            {\bf Pyplot} é uma coleção de funções no estilo de comandos que fazem a biblioteca matplotlib funcionar como o MatLab. Cada função pyplot faz alguma alteração na plotagem do gráfico.
        \end{minipage}
    };
    %------------ CAIXA PRELIMINARES ---------------------
    \node[fancytitle, right=10pt] at (box.north west) {$M/M/\infty$ Queue};
    \end{tikzpicture}

    \begin{tikzpicture}
        \node [mybox] (box){%
            \begin{minipage}{0.3\textwidth}
                Para usar a biblioteca MatPlotLib, comece importando estes módulos Python: \\
                \\
                \textit{import numpy as np} \\
                \textit{import pandas as pd} \\
                \textit{from pandas import DataFrame, Series} \\
                \textit{import matplotlib.pyplot as plt} \\
                \textit{import matplotlib} \\
                \\
                {\bf Pyplot} é uma coleção de funções no estilo de comandos que fazem a biblioteca matplotlib funcionar como o MatLab. Cada função pyplot faz alguma alteração na plotagem do gráfico.
            \end{minipage}
        };
        %------------ CAIXA PRELIMINARES ---------------------
        \node[fancytitle, right=10pt] at (box.north west) {$M/M/m$ Queue};
        \end{tikzpicture}

        \begin{tikzpicture}
            \node [mybox] (box){%
                \begin{minipage}{0.3\textwidth}
                    Para usar a biblioteca MatPlotLib, comece importando estes módulos Python: \\
                    \\
                    \textit{import numpy as np} \\
                    \textit{import pandas as pd} \\
                    \textit{from pandas import DataFrame, Series} \\
                    \textit{import matplotlib.pyplot as plt} \\
                    \textit{import matplotlib} \\
                    \\
                    {\bf Pyplot} é uma coleção de funções no estilo de comandos que fazem a biblioteca matplotlib funcionar como o MatLab. Cada função pyplot faz alguma alteração na plotagem do gráfico.
                \end{minipage}
            };
            %------------ CAIXA PRELIMINARES ---------------------
            \node[fancytitle, right=10pt] at (box.north west) {$M/G/1$ Queue};
            \end{tikzpicture}

\end{multicols*}
\end{document}
